\documentclass{report}

\input{preamble}
\input{macros}
\input{letterfonts}

\graphicspath{{./img}}

\title{\Huge{Théorie du transport optimal}}
\author{\huge{Racine Florian}}


\renewcommand{\contentsname}{Table des matières}
\renewcommand{\listfigurename}{Liste des figures}

\begin{document}

\maketitle
\newpage
\pdfbookmark[section]{\contentsname}{toc}
\tableofcontents
\pagebreak

\chapter{Introduction}
\section{Formulation du problème}
\qs{}{Quelle est la façon optimal de transporter un tas de sable dans un trou ?}
\qs{}{Comment constuire un chateau de sable d'une forme données à partir d'un tas de sable ?}

\begin{figure}[htpb]
    \centering
    \includegraphics[width=0.8\textwidth]{intro}
    \caption{Transporter un tas de sable dans un trou}
\end{figure}

\nt{ Avec le même nombre de grain de sable et la même masse.}

\chapter{Modélisation}
$\nu \in \mathcal{P}(\mathbb{R})$ ; $\mu \in \mathcal{P}(\mathbb{R})$
\dfn{}{$\forall A \in \mathcal{P} (\mathbb{R}),\mu[A]$ décrit quelle quantité de sable est dans A.
}
\begin{figure}[htpb]
    \centering
    \includegraphics[width=0.3\textwidth]{partieMu}
    \caption{$\mu[A]$}
\end{figure}

\dfn{}{
Cout infinitésimal :
\begin{displaymath}
C:
\left|
  \begin{array}{rcl}
      \mathbb{R}*\mathbb{R} & \longrightarrow & \mathbb{R} \\
      (x,y) & \longmapsto & C(x,y) \\
  \end{array}
\right.
\end{displaymath}
Cout de transporter un grain de sable de x vers y.
}
\qs{}{Comment transporter un tas de sable avec un cout global minimal ?}
\dfn{}{
Un plan de transport entre les mesures $\mu$  et  $\nu$ est une mesure de probabilité : \\
$\Pi \in \mathcal{P}(\mathbb{R}*\mathbb{R})$ à pour marginale $\mu$ et $\nu$.
}

\nt{
$\Pi \in \mathcal{P}(\mathbb{R}*\mathbb{R})$ à pour marginal $\mu$ et $\nu$ \\
$\Leftrightarrow \forall $A,B enssemble mesurable avec A$ \subset \mathbb{R}$ et B $\subset \mathbb{R}$
$\left\{
\begin{array}{l}
\Pi[A\times\mathbb{R}] = \mu[A] \\
\Pi[\mathbb{R}\times B] = \mu[B]
\end{array}
\right.$\\
$\Leftrightarrow \forall \varphi \in C^{0}(\mathbb{R}), \Psi \in C^{0}(\mathbb{R}) : $
$\displaystyle \int_{{\mathbb{R} \times \mathbb{R}}} {\varphi (x) + \Psi (y)} \: d{\Pi (x,y)}$
$= \displaystyle \int {\varphi (x)} \: d{\mu (x)}$
}

\nt{
On notera, $\Pi ( \mu , \nu )$ = \{ $\Pi \in \mathcal{P} (\mathbb{R} \times \mathbb{R}) | \Pi$
a pour marginal, $\mu, \nu$ \}\\
On remarquera que, $\Pi ( \mu , \nu ) \neq \varnothing $\\
$I[\Pi] = \displaystyle \int_{\mathbb{R}^{2}} {C(x,y)} \: d{\Pi (x,y)}$ Le cout total assocé au plan de transport optimal.\\

On cherche, $\tau_{c} (\mu , \nu ) = INF_{\Pi \in \Pi(\mu, \nu)} (I[\Pi])$
}

\dfn{}{
    S'il existe $\Pi_{0} \in \mathcal{P} ( \mathbb{R} \times \mathbb{R} )$
    tel que $I[\Pi_{0}] = \tau_{c} (\mu , \nu)$\\
    $\Pi_{0}$ est appelé un \bf{plan de transfert optimal}
}

\begin{figure}[htpb]
    \centering
    \includegraphics[width=0.4\textwidth]{exemple1}
\end{figure}
\ex{Kotorovitch}{
$a < b$\\
$c < d$\\
$C(x,y) = |x-y|^{2}$\\
$\mu = \frac{1}{2} (\delta_{a} + \delta_{b})$\\
$\nu = \frac{1}{2} (\delta_{c} + \delta_{d})$
\qs{}{
$\Pi ( \mu , \nu) = ?$
}
\sol{
$\Pi_{\alpha} = \frac{1}{2} ( \alpha \delta_{(a,c)} + (1 - \alpha ) \delta_{(a,d)} + (1 - \alpha ) \delta_{(b,c)} + \alpha \delta_{(b,d)})$
\boldmath
$\Pi(\mu,\nu) = \{\Pi_{\alpha} | \alpha \in [0, 1] \}$
\unboldmath
}
\qs{}{
    Calculer : $I[ \Pi ] \forall \Pi \in \Pi( \mu , \nu )$
}
\sol{
    $I[\Pi_{\alpha}] = \displaystyle \int_{\mathbb{R}^{2}} {C(x,y)} \: d{\Pi_{\alpha} (x,y)}$\\
    $I[\Pi_{\alpha}] = \frac{1}{2}(\alpha C(a,c) + (1 - \alpha) C(a,d) + (1 - \alpha) C(b,c) + \alpha C(b,d)) $\\
    $I[\Pi_{\alpha}] = \frac{1}{2}(a^{2}+b^{2}+c^{2}+d^{2}) - \alpha (ac + bd) -(1 - \alpha) (ad+ cb)$
}
\qs{}{
    Trouver : $\tau_{c}( \mu , \nu )$
}
\sol{
$P(\alpha) = ac - bd + ad + cb$\\
$P(\alpha) = (d-c)(a-b) < 0$\\
Donc, $P(\alpha)$ atteint son min en $\alpha = 1$\\
$\Pi_{0} = \Pi_{\alpha = 1}$\\
Donc, $a \rightarrow c$\\
et, $b \rightarrow d$
}
}
\chapter{La formulation du problème de transfert optimal de Monge}
\nt{
On autorise pas le fait de couper les masses.
A chaque x est associé une unique y.\\
On dit que T envoie $\mu$ sur $\nu$ et on note : $T \# \mu = \nu$
}
\mprop{}{
    $\forall A \subset \mathbb{R}$ partie mesurable : $\nu(A) = \mu(T^{-1}(A))$
}
$\Leftrightarrow$
\mprop{}{
    $\forall \varphi$ continue :$\displaystyle\int_{\mathbb{R}} {\varphi (y)} \: d{\nu (y)} = $
    $\displaystyle \int_{\mathbb{R}} {(\varphi o T)(x)} \: d{\mu (x)}$\\
    $\tau_{c}^{M} (\mu , \nu) = INF_{T tq T \#f = \nu} I[T]$\\
    $I(T) = \displaystyle\int_{\mathbb{R}} {C(x,T(x))} \: d{\mu (x)}$
}
\nt{
Solution de cout optimal d'après Kantorovitch $\le$ Solution de cout optimal d'après Monge\\
Dans le première exemple ils coincident.\\
}
\nt{
Kantorovitch définit un problème linéare en $\Pi$.\\
Monge définit un problème non linéare en T.
}

\nt{
    Problème de kantorovitch admet toujours une solution $\Pi_{0}$.\\
    Problème de Monge n'admet pas toujours de solution n'y même d'application qui envoi $\mu$ sur $\nu$.
}
\ex{}{
$\left\{
\begin{array}{l}
\mu \in \mathcal{P}(\mathbb{R})\\
\nu = \delta_{a}\\
\end{array}
\right.$\\
    \textbf{Kantorovitch :} $\Pi(\mu , \nu) = \{ \mu \otimes \delta_{a} \}$\\
    \textbf{Monge :} Quelles sont les T tel que $T\# \mu = \nu$ ?\\
    Il en existe une seule: 
    \begin{displaymath}
    \forall x | T:
    \left|
      \begin{array}{rcl}
        x & \longrightarrow & a \\
        \mathbb{R} & \longmapsto & \mathbb{R} \\
      \end{array}
    \right.
    \end{displaymath}
    $\tau_{c}^{M} (\mu , \nu) = \tau_{c} (\mu , \nu)$\\
    D'une part : \\
    $\tau_{c}^{M} (\mu , \nu) = \displaystyle\int_{\mathbb{R}} {C(T(x),x)} \: d{\mu (x)}$\\
    $\tau_{c}^{M} (\mu , \nu) = \displaystyle\int_{\mathbb{R}} {C(0,x)} \: d{\mu (x)}$
    D'autre part : \\
    $\tau_{c} (\mu , \nu) = \displaystyle\int_{\mathbb{R}} {C(x,y)} \: d{\Pi (x,y)}$\\
    $\tau_{c} (\mu , \nu) = \displaystyle\int_{\mathbb{R}} {C(x,y)} \: d{(\mu \otimes \delta_{a})(x,y)}$\\
    $\tau_{c} (\mu , \nu) = \displaystyle\int_{\mathbb{R}} {C(x,y)} \: d{\mu (x)} d{\delta_{a}}$\\
    $\tau_{c} (\mu , \nu) = \displaystyle\int_{\mathbb{R}} {C(x,0)} \: d{\mu (x)}$\\
}

\ex{}{
$\left\{
\begin{array}{l}
    \mu = \frac{1}{n} \sum_{i=1}^{n} \delta_{x_{i}}\\
    \nu = \frac{1}{n} \sum_{i=1}^{n} \delta_{y_{i}}\\
\end{array}
\right.$\\
Les plans de transporte $\Pi$ entre $\mu$ et $\nu$ peuvent être représenté par des matrices bistochastiques de tailles n.\\
$0 \le \Pi_{i,j} \le 1$\\
$\sum_{i=1}^{n} \Pi_{i,j} = 1$\\
$\sum_{j=1}^{n} \Pi_{i,j} = 1$\\
\nt{
On note $\mathcal{B}_{n}$ l'enssemble des matrices bisctochastiques.\\
}
Soit $\Pi \in \mathcal{B}_{n}$ : $I[\Pi] = \frac{1}{n} \sum_{i=1}^{n} C(x_{i}, y_{i}) \Pi_{i,j}$\\
$\tau_{c}(\mu ,\nu) = INF_{\Pi \in \mathcal{B}_{n}} \{\frac{1}{n}\sum_{i=1}^{n} \Pi_{i,j} C(x_{i}, y_{i})$\\
    Il s'agit d'un problème linéaire de minimisation sur un enssemble convexe.\\
}
\mprop{Enssemble convexe}{
    $\mathcal{B}_{n}$ est convexe $\Leftrightarrow A,B \in \mathcal{B}_{n}$ alors $\forall \theta \in [0,1] | \theta A + (1 - \theta ) B \in \mathcal{B}_{n}$
}
\dfn{Points extremaux}
{
    L'enssemble des points extremaux de E convexe est l'enssemble des $e \in E$ tel que :\\
    si $e=\theta e_{1} + (1 - \theta ) e_{2}$ avec $\theta \in [0, 1], e_{1} \in E, e_{2} \in E$\\
    Alors $\theta = 0$ ou $\theta = 1$
}
\thm{Théorème de Choquet}{
F est linéaire sur un domaine K convexe et compact, alors F admet au moin un minimum.
Parmi les minimums de F au moin l'un d'eux est un extrema de K.
}
\thm{Théorème de Birkhoff}{
    $\mathcal{B}_{n}$ est convexe et compact.\\
$\mathcal{B}_{n}$ admet n points extremaux qui sont les matrices de permutations\\
Ainsi, le min pour le problème de Kantorovitch est atteint pour
$\left\{
\begin{array}{l}
    \Pi_{i,j} = 1 | si j = \sigma(i)\\
    \Pi_{i,j} = 0 | sinon\\
\end{array}
\right.$\\
}
\end{document}
