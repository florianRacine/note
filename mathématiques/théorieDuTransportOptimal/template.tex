\documentclass{report}

\input{preamble}
\input{macros}
\input{letterfonts}

\graphicspath{{./img}}

\title{\Huge{Théorie du transport optimal}}
\author{\huge{Racine Florian}}


\renewcommand{\contentsname}{Table des matières}
\renewcommand{\listfigurename}{Liste des figures}

\begin{document}

\maketitle
\newpage
\pdfbookmark[section]{\contentsname}{toc}
\tableofcontents
\pagebreak

\chapter{Cours}
\section{Introduction}
\subsection{Formulation du problème}
\qs{}{Quelle est la façon optimal de transporter un tas de sable dans un trou ?}
\qs{}{Comment constuire un chateau de sable d'une forme données à partir d'un tas de sable ?}

\begin{figure}[htpb]
    \centering
    \includegraphics[width=0.8\textwidth]{intro}
    \caption{Transporter un tas de sable dans un trou}
\end{figure}

\nt{ Avec le même nombre de grain de sable et la même masse.}

\section{Modélisation}
$\nu \in \mathcal{P}(\mathbb{R})$ ; $\mu \in \mathcal{P}(\mathbb{R})$
\dfn{}{$\forall A \in \mathcal{P} (\mathbb{R}),\mu[A]$ décrit quelle quantité de sable est dans A.
}
\begin{figure}[htpb]
    \centering
    \includegraphics[width=0.3\textwidth]{partieMu}
    \caption{$\mu[A]$}
\end{figure}

\dfn{}{
Cout infinitésimal :
\begin{displaymath}
C:
\left|
  \begin{array}{rcl}
      \mathbb{R}*\mathbb{R} & \longrightarrow & \mathbb{R} \\
      (x,y) & \longmapsto & C(x,y) \\
  \end{array}
\right.
\end{displaymath}
Cout de transporter un grain de sable de x vers y.
}
\qs{}{Comment transporter un tas de sable avec un cout global minimal ?}
\dfn{}{
Un plan de transport entre les mesures $\mu$  et  $\nu$ est une mesure de probabilité : \\
$\Pi \in \mathcal{P}(\mathbb{R}*\mathbb{R})$ à pour marginale $\mu$ et $\nu$.
}

\nt{
$\Pi \in \mathcal{P}(\mathbb{R}*\mathbb{R})$ à pour marginal $\mu$ et $\nu$ \\
$\Leftrightarrow \forall $A,B enssemble mesurable avec A$ \subset \mathbb{R}$ et B $\subset \mathbb{R}$
$\left\{
\begin{array}{l}
\Pi[A\times\mathbb{R}] = \mu[A] \\
\Pi[\mathbb{R}\times B] = \mu[B]
\end{array}
\right.$\\
$\Leftrightarrow \forall \varphi \in C^{0}(\mathbb{R}), \Psi \in C^{0}(\mathbb{R}) : $
$\displaystyle \int_{{\mathbb{R} \times \mathbb{R}}} {\varphi (x) + \Psi (y)} \: d{\Pi (x,y)}$
$= \displaystyle \int {\varphi (x)} \: d{\mu (x)}$
}

\nt{
On notera, $\Pi ( \mu , \nu )$ = \{ $\Pi \in \mathcal{P} (\mathbb{R} \times \mathbb{R}) | \Pi$
a pour marginal, $\mu, \nu$ \}\\
On remarquera que, $\Pi ( \mu , \nu ) \neq \varnothing $\\
$I[\Pi] = \displaystyle \int_{\mathbb{R}^{2}} {C(x,y)} \: d{\Pi (x,y)}$ Le cout total assocé au plan de transport optimal.\\

On cherche, $\tau_{c} (\mu , \nu ) = INF_{\Pi \in \Pi(\mu, \nu)} (I[\Pi])$
}

\dfn{}{
    S'il existe $\Pi_{0} \in \mathcal{P} ( \mathbb{R} \times \mathbb{R} )$
    tel que $I[\Pi_{0}] = \tau_{c} (\mu , \nu)$\\
    $\Pi_{0}$ est appelé un \bf{plan de transfert optimal}
}


\ex{Exemple trivial (Kotorovitch)}{
$a < b$\\
$c < d$\\
$C(x,y) = |x-y|^{2}$\\
$\mu = \frac{1}{2} (\delta_{a} + \delta_{b})$\\
$\nu = \frac{1}{2} (\delta_{c} + \delta_{d})$
\qs{}{
$\Pi ( \mu , \nu) = ?$
}
\sol{
$\Pi_{\alpha} = \frac{1}{2} ( \alpha \delta_{(a,c)} + (1 - \alpha ) \delta_{(a,d)} + (1 - \alpha ) \delta_{(b,c)} + \alpha \delta_{(b,d)})$
\boldmath
$\Pi(\mu,\nu) = \{\Pi_{\alpha} | \alpha \in [0, 1] \}$
\unboldmath
}
\qs{}{
    Calculer : $I[ \Pi ] \forall \Pi \in \Pi( \mu , \nu )$
}
\sol{
    $I[\Pi_{\alpha}] = \displaystyle \int_{\mathbb{R}^{2}} {C(x,y)} \: d{\Pi_{\alpha} (x,y)}$\\
    $I[\Pi_{\alpha}] = \frac{1}{2}(\alpha C(a,c) + (1 - \alpha) C(a,d) + (1 - \alpha) C(b,c) + \alpha C(b,d)) $\\
    $I[\Pi_{\alpha}] = \frac{1}{2}(a^{2}+b^{2}+c^{2}+d^{2}) - \alpha (ac + bd) -(1 - \alpha) (ad+ cb)$
}
\qs{}{
    Trouver : $\tau_{c}( \mu , \nu )$
}
\sol{
$P(\alpha) =  \frac{\partial I[\Pi_{\alpha}]}{\partial \alpha}$
$\implies P(\alpha) = -ac - bd + ad + cb$;
$\implies P(\alpha) = (d-c)(a-b) < 0$\\
Donc, $I[\Pi_{\alpha}]$ atteint son min en $\alpha = 1$\\
$\Pi_{0} = \Pi_{\alpha = 1}$\\
Donc, $a \rightarrow c$ et, $b \rightarrow d$
}
}
\begin{figure}[htpb]
   \centering
    \includegraphics[width=0.4\textwidth]{exemple1}
    \caption{Solution}
\end{figure}
\section{La formulation du problème de transfert optimal de Monge}
\nt{
On autorise pas le fait de couper les masses.
A chaque x est associé une unique y.\\
On dit que T envoie $\mu$ sur $\nu$ et on note : $T \# \mu = \nu$
}
\mprop{}{
    $\forall A \subset \mathbb{R}$ partie mesurable : $\nu(A) = \mu(T^{-1}(A))$
}
$\Leftrightarrow$
\mprop{}{
    $\forall \varphi$ continue :$\displaystyle\int_{\mathbb{R}} {\varphi (y)} \: d{\nu (y)} = $
    $\displaystyle \int_{\mathbb{R}} {(\varphi o T)(x)} \: d{\mu (x)}$\\
    $\tau_{c}^{M} (\mu , \nu) = INF_{T tq T \#f = \nu} I[T]$\\
    $I(T) = \displaystyle\int_{\mathbb{R}} {C(x,T(x))} \: d{\mu (x)}$
}
\nt{
Solution de cout optimal d'après Kantorovitch $\le$ Solution de cout optimal d'après Monge\\
Dans le première exemple ils coincident.\\
}
\nt{
Kantorovitch définit un problème linéare en $\Pi$.\\
Monge définit un problème non linéare en T.
}

\nt{
    Problème de kantorovitch admet toujours une solution $\Pi_{0}$.\\
    Problème de Monge n'admet pas toujours de solution n'y même d'application qui envoi $\mu$ sur $\nu$.
}
\ex{}{
$\left\{
\begin{array}{l}
\mu \in \mathcal{P}(\mathbb{R})\\
\nu = \delta_{a}\\
\end{array}
\right.$\\
    \textbf{Kantorovitch :} $\Pi(\mu , \nu) = \{ \mu \otimes \delta_{a} \}$\\
    \textbf{Monge :} Quelles sont les T tel que $T\# \mu = \nu$ ?\\
    Il en existe une seule: 
    \begin{displaymath}
    \forall x | T:
    \left|
      \begin{array}{rcl}
        x & \longrightarrow & a \\
        \mathbb{R} & \longmapsto & \mathbb{R} \\
      \end{array}
    \right.
    \end{displaymath}
    $\tau_{c}^{M} (\mu , \nu) = \tau_{c} (\mu , \nu)$\\
    D'une part : \\
    $\tau_{c}^{M} (\mu , \nu) = \displaystyle\int_{\mathbb{R}} {C(T(x),x)} \: d{\mu (x)}$\\
    $\tau_{c}^{M} (\mu , \nu) = \displaystyle\int_{\mathbb{R}} {C(0,x)} \: d{\mu (x)}$\\
    D'autre part : \\
    $\tau_{c} (\mu , \nu) = \displaystyle\int_{\mathbb{R}} {C(x,y)} \: d{\Pi (x,y)}$\\
    $\tau_{c} (\mu , \nu) = \displaystyle\int_{\mathbb{R}} {C(x,y)} \: d{(\mu \otimes \delta_{a})(x,y)}$\\
    $\tau_{c} (\mu , \nu) = \displaystyle\int_{\mathbb{R}} {C(x,y)} \: d{\mu (x)} d{\delta_{a}}$\\
    $\tau_{c} (\mu , \nu) = \displaystyle\int_{\mathbb{R}} {C(x,0)} \: d{\mu (x)}$\\
}

\ex{}{
$\left\{
\begin{array}{l}
    \mu = \frac{1}{n} \sum_{i=1}^{n} \delta_{x_{i}}\\
    \nu = \frac{1}{n} \sum_{i=1}^{n} \delta_{y_{i}}\\
\end{array}
\right.$\\
Les plans de transports $\Pi$ entre $\mu$ et $\nu$ peuvent être représenté par des matrices bistochastiques de tailles n.\\
$0 \le \Pi_{i,j} \le 1$\\
$\sum_{i=1}^{n} \Pi_{i,j} = 1$\\
$\sum_{j=1}^{n} \Pi_{i,j} = 1$\\
\nt{
On note $\mathcal{B}_{n}$ l'enssemble des matrices bisctochastiques.\\
}
Soit $\Pi \in \mathcal{B}_{n}$ : $I[\Pi] = \frac{1}{n} \sum_{i=1}^{n} C(x_{i}, y_{i}) \Pi_{i,j}$\\
$\tau_{c}(\mu ,\nu) = INF_{\Pi \in \mathcal{B}_{n}} \{\frac{1}{n}\sum_{i=1}^{n} \Pi_{i,j} C(x_{i}, y_{i})$\\
    Il s'agit d'un problème linéaire de minimisation sur un enssemble convexe.\\
}
\mprop{Enssemble convexe}{
    $\mathcal{B}_{n}$ est convexe $\Leftrightarrow A,B \in \mathcal{B}_{n}$ alors $\forall \theta \in [0,1] | \theta A + (1 - \theta ) B \in \mathcal{B}_{n}$
}
\dfn{Points extremaux}
{
    L'enssemble des points extremaux de E convexe est l'enssemble des $e \in E$ tel que :\\
    si $e=\theta e_{1} + (1 - \theta ) e_{2}$ avec $\theta \in [0, 1], e_{1} \in E, e_{2} \in E$\\
    Alors $\theta = 0$ ou $\theta = 1$
}
\thm{Théorème de Choquet}{
F est linéaire sur un domaine K convexe et compact, alors F admet au moin un minimum.
Parmi les minimums de F au moin l'un d'eux est un extrema de K.
}
\thm{Théorème de Birkhoff}{
    $\mathcal{B}_{n}$ est convexe et compact.\\
$\mathcal{B}_{n}$ admet n points extremaux qui sont les matrices de permutations\\
Ainsi, le min pour le problème de Kantorovitch est atteint pour
$\left\{
\begin{array}{l}
    \Pi_{i,j} = 1 | si j = \sigma(i)\\
    \Pi_{i,j} = 0 | sinon\\
\end{array}
\right.$\\
}

\section{La dualité de Kantorovitch}
\subsection{La théorie}%

\thm{Dualité de Kantorovitch}{
    $\mu \in \mathcal{P} (\mathbb{R}^{n})$\\
    $\nu \in \mathcal{P} (\mathbb{R}^{n})$\\
    C semi continue inférieurement ( par exemple C continue)
    Soit $(\varphi, \psi ) \in \phi_{c}$\\
    $I(\varphi, \psi) = \displaystyle \int_{\mathbb{R}^{n}} {\varphi(x)} \: d{\mu (x)}$
    $+ \displaystyle \int_{\mathbb{R}^{n}} {\psi(y)} \: d{\nu (y)}$\\
    $\phi_{c} = \{ (\varphi, \psi) \in C^{0}$ tq $\varphi (x) + \psi(y) \ge C(x,y)$ presque partout $\}$
    Alors, $INF_{\Pi \in \Pi(\mu)}I[\Pi] = SUP_{\varphi , \psi} J(\varphi , \psi)$
}
\nt{
    Interprétation :\\
    On embauche un transporteur.\\
    Il achète de la masse située en x au pris $\varphi(x)$.\\
    Il vous débarasse au prix $\displaystyle \int {\varphi(x)} \: d{\mu(x)}$\\
    Il vous revend de la masse en y au prix $\psi(y)$\\
    On rachète $\nu$ au prix $\displaystyle \int {\psi(y)} \: d{\nu(y)}$ 
}
\begin{myproof}
    Soit $(\varphi , \psi ) \in \phi_{c}$\\
    Soit $\Pi \in \Pi(\mu, \nu) | I[\Pi] ) \tau(\mu , \nu)$

    $J(\varphi, \psi) = \displaystyle \int_{\mathbb{R}^{n}} {\varphi(x)} \: d{\mu (x)}$
    $+ \displaystyle \int_{\mathbb{R}^{n}} {\psi(y)} \: d{\nu (y)}$\\
    $\ge \displaystyle \int {C(x,y)} \: d{\Pi(x,y)}$
    $= INF_{\Pi} I[\Pi] = \tau(\mu,\nu)$\\
    Ainsi, $J(\varphi , \psi) \ge INF_{\Pi} I[Pi]$
    
    
\end{myproof}

\dfn{Les fonction C concaves}{
    Soit, $\varphi : \mathbb{R}^{n} \rightarrow \mathbb{R} \cup \{ -\infty \}$\\
    On définit sa fonction C-conjuguée par :
    \begin{displaymath}
    \varphi^{c}:
    \left|
      \begin{array}{rcl}
        \mathbb{R}^{n} \ & \longrightarrow & \mathbb{R} \cup - \infty \\
        x & \longmapsto & a \\
      \end{array}
    \right.
    \end{displaymath}
}

\mlenma{}{
lemme
}

\thm{}{
    La dualité de Kantorovitch peut être restreinte à des couples de fonction C-conjuguées.\\
    $SUP_{(\varphi , \psi) \in (C(\mathbb{R}^{n})^{2}} J(\varphi, \psi) = MAX_{(\psi^{c} , \psi)} J(\psi^{c} , \psi)$
}

\begin{myproof}
    On montre que le sup est un max.
\end{myproof}

\cor{Les plans de transferts optimals sont caractérisés par leur support}{
    Si $(\varphi , \psi)$ est un maximiseur pour le problème de Kantorovitch dual, alors\\
    $\Pi \in \Pi(\mu , \nu)$ est un minimiseur pour le problème de Kantorovitch primal si et seulement si $\Pi$ est concentrée sur
    $\{ (x,y) \in \mathbb{R}^{n} \times \mathbb{R}^{n} | \varphi (x) + \psi(y) = C(x,y) \}$
}

\subsection{Appliquation(s)}
\mprop{
On considère un objet indexé par j présent en quantité $\nu_{j}$.\\
(type(j), quantjté(j)) = (j, $\nu_{j}$)\\
On considère un consomateur indexé par i présent en quantité $\mu_{i}$.\\
(type(i), quantité(i)) = (i, $\mu_{i}$)\\
Utilité de l'objet j pour l'agent i.\\
\textbf{Hypothèse} L'utilité est tranférable.\\
L'objet j a une utilité nette $U_{r,j} - P$.\\
Pour un système de prix $P_{j}$, l'agent i choisit l'objet $j_{p}$ qui maximise  $U_{ij} - P_{j}$.\\
Transfert optimal: sup $\sum_{ij} U_{ij} \Pi_{ij}$ sous la contrainte $\sum_{j}  \Pi_{ij} = \mu_{i}$ $; \sum_{i}  \Pi_{ij} = \nu_{j}$ 
}

\nt{
    \textbf{Explication :}\\
    $\mu = \sum_{i} \mu_{i} \delta_{x_{i}}$\\
    $\nu = \sum_{j} \nu_{j} \delta_{y_{j}}$\\
    $C(i,j) = - u_{i,j}$\\
    $I[\Pi] = - \sum_{ij} u_{ij} \Pi_{ij}$\\
    \textbf{Utilité maximale :} $INF_{\Pi} I[\Pi] = SUP \{ - \sum_{i} \varphi_{i} \mu_{i} - \sum_{j} \psi_{j} \nu{j} \}$\\
    \textbf{Problème dual :} $(D) : INF_{P_{j}} \{ \sum_{j} \nu_{j} P_{j} + \sum_{i} \mu_{i} MAX_{j} (\nu{ij} - P{j}) \}$
}

\dfn{Prix d'équilibre}{
    Un système de prix qui satisfait (D) est un prix d'équilibre du problème.
    Un tel système de prix permet d'atteindre l'optimum global
}

\chapter{TD}
\section{TD1}

\exr{
\textbf{Exercice 1.} On considère le coût $c(x, y)=|x-y|$. Dans chacun des cas, donner les solutions des problèmes de Monge et de Kantorovitch.\\
1. $\mu=\frac{1}{2} \delta_0+\frac{1}{2} \delta_1, \quad \nu=\frac{1}{3} \delta_{-1}+\frac{1}{3} \delta_2+\frac{1}{3} \delta_3$,\\
2. $\mu=\frac{1}{2} \delta_0+\frac{1}{2} \delta_1, \quad \nu=\frac{1}{2} \delta_0+\frac{1}{2} \delta_1$,\\
3. $\mu=\frac{1}{3} \delta_0+\frac{1}{3} \delta_1+\frac{1}{3} \delta_2, \quad \nu=\frac{1}{3} \delta_{-1}+\frac{1}{3} \delta_0+\frac{1}{3} \delta_3$.\\
}
\sol{
1)\\
$0 \ge \alpha \ge \frac{1}{3}$ \\
$0 \ge \beta \ge \frac{1}{3}$ \\
$\frac{1}{6} \ge \alpha + \beta \ge \frac{1}{2}$ \\
$\Pi_{\alpha , \beta} = \alpha \delta_{(0,-1)} + \beta \delta_{(0,2)} + (\alpha + \beta ) \delta_{(0,3)}$

2)\\
Equivalent à l'exemple du cours.

3)
}
\exr{
    \textbf{Exercice 2.} Soit $T: \mathbb{R} \rightarrow \mathbb{R}$ définie par $T(x)=x+1, S: \mathbb{R} \rightarrow \mathbb{R}$ définie par $S(x)=2 x$ et $Z: \mathbb{R} \rightarrow \mathbb{R}$ définie par $Z(x)=2-x$. On définit $\mu=\mathbb{1}_{[0,1]}$ et $\nu=\mathbb{1}_{[1,2]}$. A-t-on $T \# \mu=\nu$ ? $S \# \mu=\nu ? Z \# \mu=\nu ?$
}
\sol{
    $\displaystyle \int_{[0,1]} {(\varphi o S)(x)} \: d{\mu (x)} = $
    $\displaystyle\int_{[0,1]} {\varphi (2x)} \: d{x} = $
    $\frac{1}{2} \displaystyle \int_{[0,2]} {\varphi(y)} \: d{y}$

    $ \Rightarrow S \# \mu = \frac{1}{2} \mathbb{1}_{[0,2]} dx$\\

    $\displaystyle \int_{[0,1]} {(\varphi o T)(x)} \: d{\mu (x)} = $
    $\displaystyle\int_{[0,1]} {\varphi (1+x)} \: d{x} = $
    $\displaystyle\int_{[0,2]} {\varphi (y)} \: d{y} = $
    $\displaystyle\int {\varphi (y)} \: d{\nu (y)}$

    $ \Rightarrow T \# \mu = \nu$
}

\exr{
    \textbf{Exercice 3.} (Non-unicité pour un coût convexe - Book shifting). On définit $\mu=\mathbb{1}_{[0,2]}$ et $\nu=\mathbb{1}_{[1,3]}$ et le coût $c(x, y)=|x-y|$. Soit $T_1(x)=x+1$ et
$$
T_2(x)=\left\{\begin{array}{l}
x+2, \text { si } x \in[0,1], \\
x, \text { si } x \in(1,2] .
\end{array}\right.
$$
Montrer que $T_1$ et $T_2$ sont deux applications optimales.
}
\sol{
Vérifiez que :\\
$T_{1} \# \mu = \nu$ \\
$T_{2} \# \mu = \nu$ \\
    $I(T_{1}) = \displaystyle\int_{\mathbb{R}} {C(x,T_{1}(x))} \: d{\mu (x)} = 1$\\
    $I(T_{2}) = \displaystyle\int_{[0,2]} {C(x,T_{2}(x))} \: d{\mu (x)}$\\
    $I(T_{2}) = \frac{1}{2} \displaystyle\int_{[0,1]} {|x-T_{2}(x)|} \: d{x} + 0 = 1 $\\
    $I(T) \ge |\displaystyle\int {|x|} \: d{\mu(x)} - \displaystyle\int {|T(x)|} \: d{\mu(x)}|$\\
    $I(T) = 1$
}


\exr{
    \textbf{Exercice 4.} (Non existence d'une application de transport).
On prend $\mu$ la mesure uniforme sur $[0,1]$ et $\nu$ la mesure uniforme sur $[-1,1]$. On considère le coût $c(x, y)=\left(x^2-y^2\right)^2$.
1. Pour tout entier $n$ on définit l'application
$$
T_n(x)=\left\{\begin{array}{l}
2 x-\frac{k}{2 n}, \quad \text { pour } x \in\left[\frac{k}{2 n}, \frac{k+1}{2 n}\right] \text { si } k \text { est pair, } \\
-2 x+\frac{k+1}{2 n}, \quad \text { pour } x \in\left[\frac{k}{2 n}, \frac{k+1}{2 n}\right] \text { si } k \text { est impair. }
\end{array}\right.
$$
Monter que $T_n \# \mu=\nu$ et montrer que
$$
\lim _{n \rightarrow \infty} \int_0^1 c\left(x, T_n(x)\right) d \mu(x)=0 .
$$
2. En déduire qu'il n'esxite pas d'application de transport qui soit optimale.
3. Construire un plan de transport optimal.
}
\exr{
\textbf{Exercice 5.} (Transport quadratique et translation).
On considère le coût $c(x, y)=(x-y)^2$ sur $\mathbb{R}^2$ Pour $a \in \mathbb{R}$, on définitit la translation $\tau_a(x)=x-a$. Soit $f$ et $g$ deux fonctions continues Le but est de montrer que
$$
\mathcal{T}_c\left(f \circ \tau_a, g \circ \tau_b\right)=\mathcal{T} c(f, g)+(b-a)^2+2(b-a)\left(m_g-m_f\right),
$$
où
$$
m_f=\int_{\mathbb{R}} x f(x) d x, \quad m_g=\int_{\mathbb{R}} x g(x) d x .
$$
1. Soit $T$ une application optimale qui envoie $f$ sur $g$. On définit $S$ par $S(x)=T(x-a)+b$. Montrer que $S \#\left(f \circ \tau_a\right)=g \circ \tau_b$.
2. Montrer que
$$
\left.\mathcal{T}_c\left(f \circ \tau_a, g \circ \tau_b\right) \leq \int_{\mathbb{R}}|S(x)-x|^2 f\left(\tau_a(x)\right)\right) d x
$$
3. En déduire que
$$
\mathcal{T}_c\left(f \circ \tau_a, g \circ \tau_b\right) \leq \mathcal{T} c(f, g)+(b-a)^2+2(b-a)\left(m_g-m_f\right),
$$
4. De même montrer que
$$
\mathcal{T}_c\left(f \circ \tau_a, g \circ \tau_b\right) \geq \mathcal{T} c(f, g)+(b-a)^2+2(b-a)\left(m_g-m_f\right),
$$
et en déduire (1)
5. En déduire que $\mathcal{T}_c\left(\mathbb{1}_{[0,1]}, \mathbb{1}_{[1,2]}\right)=1$.
}
\exr{
\textbf{Exercice 6.} (Non unicité des potentiels de Kantorovitch). Montrer que si $(\varphi, \psi)$ est une paire optimale de potentiel de Kantorovitch, alors, pour tout $a \in \mathbb{R}$, la paire $(\varphi+a, \psi-a)$ l'est aussi.
}
\exr{
\textbf{Exercice 7.} (Potentiels de Kantorovitch). Donner au moins une paire optimale de potentiel de Kantorovitch pour le book-shifting problem.
}
\exr{
\textbf{Exercice 8.} Pour $x \in \mathbb{R}$ et $y \in \mathbb{R}$, on considère le coutt $c(x, y)=\Psi(|x-y|)$ avec
$$
\Psi(z)= \begin{cases}1-z, & 0 \leq z \leq 1 \\ z-1, & z \geq 1 .\end{cases}
$$
On considère les mesures $\mu=\frac{1}{2}\left(\delta_{-1}+\delta_2\right)$ et $\nu=\frac{1}{3}\left(\delta_{-1}+\delta_0+\delta_1\right)$. Calculer le coût global du transport optimal entre $\mu$ et $\nu$, et donner l'ensemble des plans de transport optimaux.
}
\end{document}
