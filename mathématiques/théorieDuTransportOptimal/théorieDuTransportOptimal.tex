%----------------------------------------------------------------------------------------
%	PACKAGES AND OTHER DOCUMENT CONFIGURATIONS
%----------------------------------------------------------------------------------------
\documentclass[11pt, a4paper]{article}

%\usepackage[french]{babel}
%\usepackage{mathpazo} % Use the Palatino font
%\usepackage{booktabs} % Required for better horizontal rules in tables
\usepackage{float} % Required for including images
\usepackage{geometry} %Required for changing the size of the marge
\usepackage[utf8]{inputenc} % Required for inputting international characters
\usepackage[T1]{fontenc} % Output font encoding for international characters
\usepackage{graphicx} % Required for including images
\usepackage{listings} % Required for insertion of code
\usepackage{enumerate} % To modify the enumerate environment
\usepackage{xcolor} % Required to add color in text
\usepackage{titling} % Easy access to assignement information
\usepackage{tabularx}
\usepackage{eso-pic}
\usepackage{caption}
\usepackage{subcaption}
\captionsetup{compatibility=false}
\usepackage{wallpaper}
\usepackage{enumitem}
\usepackage{fancyhdr}
\usepackage{hyperref}
\usepackage{amssymb}
\usepackage{mathtools}

%----------------------------------------------------------------------------------------
%	DEFINE SYNTAX COLOR TO OUTPUT CODE
%----------------------------------------------------------------------------------------

\definecolor{codegreen}{rgb}{0,0.6,0}
\definecolor{codegray}{rgb}{0.5,0.5,0.5}
\definecolor{codepurple}{rgb}{0.58,0,0.82}
\definecolor{backcolour}{rgb}{0.95,0.95,0.95}

\lstdefinestyle{mystyle}{
backgroundcolor=\color{backcolour},   
commentstyle=\color{codegreen},
keywordstyle=\color{magenta},
numberstyle=\tiny\color{codegray},
stringstyle=\color{codepurple},
basicstyle=\ttfamily\footnotesize,
breakatwhitespace=false,         
breaklines=true,                 
captionpos=b,                    
keepspaces=true,                 
numbers=left,                    
numbersep=5pt,                  
showspaces=false,                
showstringspaces=false,
showtabs=false,                  
tabsize=2
}

\lstset{style=mystyle}

%----------------------------------------------------------------------------------------
%	ASSIGNMENT INFORMATION
%----------------------------------------------------------------------------------------
\author{RACINE Florian}
\title{Théorie du transport optimal}
\geometry{hmargin=2.5cm,vmargin=1.5cm}
\pagestyle{fancy}

\fancyhead{}
\fancyfoot{}
\fancyhead[L]{\textbf{\thepage} \; Chapitre \leftmark}

%----------------------------------------------------------------------------------------
%	DEFINE COMMANDS
%----------------------------------------------------------------------------------------
\renewcommand{\contentsname}{Table des matières}
\renewcommand{\listfigurename}{Liste des figures}
\newcommand{\HRule}{\rule{\linewidth}{0.5mm}}

\begin{document}
%----------------------------------------------------------------------------------------
%	COVER PAGE
%----------------------------------------------------------------------------------------
\maketitle % Output the assignment title, created automatically using the information in the custom commands above
\vspace*{\stretch{1}}
\begin{center}
\HRule \\[0.4cm]
{ \huge \bfseries \thetitle \\[0.4cm] }
\HRule \\[2cm]
\graphicspath{{/home/evo/Pictures/logo/}}
\includegraphics[width = 7cm]{logo}
\end{center}
\vspace*{\stretch{1}}
\newpage

%----------------------------------------------------------------------------------------
%	TABLE OF CONTENTS
%----------------------------------------------------------------------------------------
\tableofcontents
\newpage

\graphicspath{{image/}} % Changing the path to include image

%----------------------------------------------------------------------------------------
%	SECTION 1
%----------------------------------------------------------------------------------------
\section{Introduction}

\subsection{Formulation du problème}

\begin{itemize}
    \item Quelle est la façon optimal de transporter un tas de sable dans un trou ?
    \item Comment constuire un chateau de sable d'une forme données à partir d'un tas de sable ?        
\end{itemize}

\section{Modélisation}
\paragraph{}
$\nu \in \mathcal{P}(\mathbb{R})$ ; $\mu \in \mathcal{P}(\mathbb{R})$

\vspace{2mm}
\fbox{\begin{minipage}{40em}
\vspace{3mm}
\paragraph{Définition :}
$\forall A \in \mathcal{P} (\mathbb{R}),\nu[A]$ décrit quelle quantité de sable est dans A.
\vspace{3mm}
\end{minipage}}
\vspace{5mm}

\vspace{2mm}
\fbox{\begin{minipage}{40em}
\vspace{3mm}
\paragraph{Définition : Cout Infinitesimal}
\begin{displaymath}
C:
\left|
  \begin{array}{rcl}
      \mathcal{R}*\mathcal{R} & \longrightarrow & \mathcal{R} \\
      (x,y) & \longmapsto & C(x,y) \\
  \end{array}
\right.
\end{displaymath}
\vspace{3mm}
\end{minipage}}
\vspace{5mm}

Cout de transporter un grain de sable de x vers y.

\paragraph{Problème :}
Comment transporter un tas de sable avec un cout global minimal ?


\vspace{2mm}
\fbox{\begin{minipage}{40em}
\vspace{3mm}
\paragraph{Définition :}
    
Un plan de transport entre les mesures $\mu$  et  $\nu$ est une mesure de probabilité : \\
$\Pi \in \mathcal{P}(\mathcal{R}*\mathcal{R})$ à pour marginale $\mu$ et $\nu$.

\vspace{3mm}
\end{minipage}}
\vspace{5mm}

\paragraph{Rappel :}
$\Pi \in \mathcal{P}(\mathcal{R}*\mathcal{R})$ à pour marginal $\mu$ et $\nu$ 
\vspace{5mm}

$\Leftrightarrow \forall $A,B enssemble mesurable avec A$ \subset \mathcal{R}$ et B $\subset \mathcal{R}$
$\left\{
\begin{array}{l}
\Pi[A\times\mathcal{R}] = \mu[A] \\
\Pi[\mathcal{R}\times B] = \mu[B]
\end{array}
\right.$
\end{document}
